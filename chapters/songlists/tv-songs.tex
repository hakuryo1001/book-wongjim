\begin{itemize}
\item 黛綠年華
\item 惜別
君心喚我心 始知相愛深
萬種恩義千般愛 未知別離恨

昔日與君 與君心印心
萬種恩義千般愛 莫教別離恨

相思呀 苦滋味 心中掛念君
憶君呀 芳心盡碎 滿襟血淚痕

今日與君 與君千里分
萬種恩義千般愛 剩得別離恨
萬種恩義千般愛 剩得別離恨



\item 花含笑

花含笑,玉生香,
幾世修得有情郎, 
郎有痴心我多情義,
從今相依誠心相向。 
飄泊身世奴心苦,
一生渴望守身以待有情郎,
雀兒唱,樂聲響,酒滿金樽客滿堂,
同醉一杯百年釀,銀鐲高燒情歌輕唱。
家國遭亂世勞心苦,今生有幸委身以事有情郎,
花前笑,步生香,
歡笑相看倆情長,緣訂今生結鸞鳳,
從今相依誠心相向,從今相依誠心相向。

\item 思君(原曲:紅豆詞)
今日我心思君君知否
枉有相思空寄付紅豆.
那堪冷清寂寥亂我心
怕對孤獨淒涼時候
忍看此際鏡中花容瘦
忍見早晚照影雙眉皺
心哀傷 心悲懮
此恨綿綿 此怨幽幽
啊 啊
剪不斷 理還亂紛紛
恰似一江春水悠悠


\item 長相思(原曲:Scarborough Fair)
令我苦痛又悲傷
天天不見君你面
長相思最斷人腸

愁眉深鎖淚眼茫茫
令我苦痛又悲傷
花開等到花已謝
長相思最斷人腸

何時君歸伴我旁
為我解去痛苦創傷
春天等到冬天雪落
長相思最斷人腸

何時君歸伴我旁
為我解去痛苦創傷
春天等到冬天雪落
長相思最斷人腸


\item 當你見到天上星星
當你見到天上星星 可有想起我
可有記得當年我的臉 曾為你更比星星笑得多
當你記起當年往事 你又會如何
可會輕輕淒然歎喟 懷念我在你心中照耀過

我像那銀河星星 讓你默默愛過
—更讓那柔柔光輝 為你解痛楚

當你見到光明星星 請你想想起我
當你見到星河燦爛 求你在心中記住我

\item 好想忘記你
本應好想忘記你
 一於由你生死
 偏偏懷念不忘
 天天仍是想你
 不知點樣忘記你
 點解成日想起
 天天懷念不忘
 心中回味不已
 記住你為我
 為我心癡
 記住你令我
 天天好歡喜
 如果心中忘記你
 等於忘記歡喜
 只好無限關懷
一生成日想你


\item 隋唐風雲
隋煬昏暴, 更遇天災, 美好山河竟變苦海, 庶民逢禍害,
齊心協力, 抗暴驅災, 拚抛鮮紅血與骸, 英雄豪氣在,
楊花落, 李花開, 引出風塵三俠來, 美談留光彩,
文傳佳藝, 武服海外, 展開中華德與才, 光芒傳萬代!


\item 射鵰英雄傳 (誰是大英雄)
絕招 好武功 十八掌一出力可降龍
大顯 威風 男兒到此是不是英雄
誰是大英雄 (大英雄)
射雕 彎鐵弓 萬世聲威震南北西東
偉績 豐功 男兒到此是不是英雄
誰是大英雄 (大英雄)
一陽指 蛤蟆功 東邪西毒南帝北丐中神通
好郭靖 俏黃蓉 誰人究竟是大英雄
練得堅忍 大勇止干戈永不居功
義氣衝霄漢 立地頂天是大英雄
才是大英雄

\item 神鵰俠侶
(女聲合唱)神鵰俠侶 英名世垂
(男獨) 問世間幾許 俠客胸襟熱血之軀
問世間多少 情人到死誓相隨 能死生相許
(女聲合唱)神鵰 俠侶 啊~啊~啊~
(男獨) 神鵰 俠侶 千秋萬世推許 啊~啊~
楊過 小龍女 到死誓相隨 願死生相許
(女聲合唱)啊~啊~啊~啊~啊~啊~
(男女合唱)神鵰俠侶 千秋萬世推許 楊過 小龍女
到死誓相隨 行俠仗義名垂

\item 相愛不相聚
春花謝 苦相思
難忘你的恩義
夢到幽谷絕嶺
雙雙訴心事

休相問 別了幾多載
難尋你所在
就算天荒地老
心中永等待

憶花下 相偎倚
為何蝶去花似淚
恨那春歸蝶去
相愛不相聚


\item 好哥哥
\item 狂潮

是他也是你和我 同相親相愛也相爭
大家偶遇在人海 你我各留痕
幾許歡與笑 多少愛和恨
那狂潮捲起 燦爛又繽紛

是苦也是甜美 人生的喜惡怎麼分
大家各自尋找 你我心中印
幾許哭與嘆 多少假與真
讓狂潮起跌 混合愛和恨

是他也是你和我 同悲歡喜惡過一生

\item 近代豪俠傳

英雄好漢 胸懷大義 五湖七海里
俠客多君子 奸徒披靡 鬼神避忌
凜然肝膽照 俠客重仁義 拳風鎮住邪惡類
大刀劈盡無理事 豪氣壓絕強淚
伏霸揚正義 當仁拋小我 捨身成大事
永垂青史裡 熱血是男兒 當仁拋小我
捨身成大事 永垂青史裡 熱血是男兒
永垂青史裡 熱血是男兒

\item 紅樓夢中我和你
夢也辛酸夢也荒唐
你我夢裡滋味
願我不失願你莫忘
此中那纏綿意
夢裡悲歡夢裡苦甜
你會明其中味

為你痴痴夢裡情懷
點點也是誠摯
寶黛凝淚血
是你和我的悲
紅樓夢中那寶黛盟
喜中有我和你

莫失莫忘
你我夢裡滋味
莫拋開
更莫棄莫離
此中你我情意

\item 黛玉葬花

落花滿天又滿煙
 花樹分離惜別時
 春花今謝明還發
 他日誰來愛花癡
 落花滿天淚滿煙
 花謝香沉春盡時
 今朝花謝儂來葬
 他日誰人葬花癡
 儂今葬花人笑癡
 他年葬儂知是誰
 一朝春盡紅顏老
 花落人亡兩不知
 一朝春盡殘紅碎
 知是紅顏老死時

\item 警幻曲

\item 晴雯撕扇

旦: (天堂歌)
心中怨憤,
受襲人疑忌,
佢太過尖酸,
恃嬌倚勢,
令人愁慮,
意態昏昏,
滿腔鬱氣,
夜來無事,
怡紅抖睡
 
生: (靜悄悄)
我薄醉欲睡,
薄醉欲睡,
意迷迷,
步轉怡紅院裡
 
(收掘白)
哦,做乜襲人瞓喺呢處呀,襲人姐,我昨日踢親你,而家重見唔見痛唧嚇,
 
旦: (白)
有心咯,二爺,
 
生: (白)
哦,乜原來系晴雯你咩
 
旦: (白)
之唔係我,哈,寶二爺對待襲人都幾細心嘅噃,
 
生: (白)
嘻嘻,對你又一樣咁細心唧,
 
旦: (白)
系咩,
 
(中板)
寶二爺,
睇你心細如塵,
不愧是個多情種子;
一返到怡紅院內,
便要向襲人慰問,
都可惜而家你問錯、問錯咗隔離。
雖則你咁細心,
但係呢個蠢鈍嘅晴雯,
 
(滾花)
邊處有襲人咁好福氣;
 
生: (白)
阿晴雯姐,
 
(長二王)
你無謂語語相譏,
做乜學成咁小器,
大觀園內,
個個都係可人兒,
我一視同仁,
無分彼此,
亦無分貴賤,
姐你莫亂想胡思。
 
(小曲燭影搖紅)
你更莫頑皮、頑皮唔講理,
一時不適意,
即刻發嬌癡,
皺起黛眉,
最令我難為,
 
旦: (接唱)
你心腸有否偏私,
二爺你自知,
因乜對待別人,
得咁細微
 
(即轉反線中板)
你無謂假細心,
既欲急於找襲人,
重何須喺處咁贅氣;
更恐防,
被人睇見,
又話我搬弄是非。
 
生: (接唱反線中板)
姐你莫嬌嗔,
其實我對你晴雯,
另有一番心意;
趁此晚風涼,
我地何妨並坐,
共訴心底衷詞。
 
旦: (三腳凳)
乜嘢叫做衷詞,無非系花言巧語咋;
我喺處坐嚇都經已過份,重點敢坐晌你隔籬。
 
生: (三腳凳)
你話不配坐在怡紅,何以在此來抖睡;
你分明有妒意,唔係點會講得咁離奇。
 
(花)
我都話過貴賤不分,你又何須顧忌;
 
(收掘白)
嚟喇嚟喇,埋嚟坐喺處傾嚇偈罷喇,
 
旦: (口古)
都係唔好勒二爺,我重要去沖涼添,系喇,今日鴛鴦送咗的水果嚟,不如我叫人拈出嚟畀阿二爺你解醉哩;
 
生: (口古)
我唔駛,我唔駛人攞畀我,我要你攞畀我食,至覺得你心事細微吖嘛。
 
旦: (白)
我重敢咩,
 
(二王)
果子放在水晶缸,
水晶缸又咁名貴,
我恐防會打爛,
個陣便罪大難填。
記否我偶爾跌斷你扇兒,
你便非常咁生氣;
粗心浮氣,
不配做你嘅侍兒。
 
生: (南音)
就算責罵幾聲呢,
何必生氣;
無非今時藉意發嬌癡。
只要博得姊你歡心,由你鍾意;
我順從你所欲,總要你樂怡怡。
 
旦: (白)
阿二爺你講的說話有冇呃我嘅唧,若果真係冇呃我嘅,嗱,我而家鍾意聽撕扇嘅聲,你畀把扇我撕睇你肯唔肯呀,
 
生: (白)
哦,得得得,你撕呀嗱,有乜所謂唧,攞畀你喇,
 
旦: (撕扇介白)
嘻嘻,哈哈,好開心,
 
生: (白)
哈哈,撕得好呀,
 
(蕩舟)
樂無限,
兩偎倚,
 
旦: (接唱)
我今撕扇求自娛,
 
生: (接唱)
一笑千金不吝計,
樂怡怡,
 
生、旦: (合唱)
一朝雲霧散,
盡去前嫌莫再提


\item 碧血劍

拋碧血誓保山河國土
揮寶劍力殲惡匪邪徒
扶持弱幼  維揚武道
奸鬼避開  衰神卻步
浩氣比天高

痴心兒女痴情事
忠心俠客忠烈圖
誠和愛  仁和義
千秋萬載  得人敬慕
萬世也敬慕

位位名家  江湖豪俠
招招武功 四海獨步
袁承志  鋤強除暴
金蛇郎君  英雄志氣豪

拋碧血誓保山河國土
揮寶劍力殲惡匪邪徒
扶持弱幼  維揚武道
奸鬼避開  衰神卻步
浩氣比天高

\item 浪淘沙
浪淘盡了千千沙
也留下了種種恨

浪留下了千百印
印中印千百心

浪淘盡了千千沙
卻留我種種恨

浪連石也沖碎
偏沖不淡我心

浪花求你來淘盡我剩下的怨恨
浪花留下那笑那恩那心
留下苦痛舊痕

~~~

浪淘淨了海中沙
快淘我心中恨
讓我在心裡
將他的笑重尋

浪花求你來淘盡我剩下的怨恨
浪花留下那笑那恩那心
留下苦痛舊痕

~~~

浪淘淨了海中沙
快淘我心中恨
讓我在心裡將他的笑重尋

\item 鑄情

\item 家變

知否世事常變 變幻原是永恆
此中波浪起跌 當然有幸有不幸
不必怨世事變 變幻才是永恆
經得風浪起跌 必將惡運變好運
月缺後月重圓 缺後月重圓 始終都會相對襯
人間的波折 經得起挫折 始終都會不枉此生
迎接那變幻 今生與你擁抱著永恆

\item 無名英雄
你記得 我也記得
世間的英雄義士
唯是那無名俠客
誰個知當年逸事

赴湯蹈烈火
為這人間添上忠義
永不求聞逹
成人捨已 行俠仗義

這許多無名人士
留下事蹟光耀百世
為後世留仁與義
萬世千秋見浩氣


\item 大亨
他也在找 我也在找 找到名利幾多
他拚命追 我拚命追 追到又如何

追到什麼 找到什麼 收到又幾多
得了什麼 失了什麼 可有認真算過

何必呢 何必呢 可知一切他朝都會身外過
得的多 還失的多 升得高的終於都會低墮

何必呢 何必呢 拋開一切束縛身心韁鎖
且向心內 仔細追尋 找那安然嘅我

找那安然嘅我 找那安然嘅我

\item 人海奇譚
生仔與生女有精或笨
皆因父母大眾不同人
千般姿態幾多古怪
人海各浮沈

好的與醜的兩相會聚
因此人世就最多奇聞
此長彼退 好醜相對
人生路自尋

一樣米養百樣人
人人都有份
千條去路點行法
人人問吓心

雖則美醜好似天預定
好壞善惡實在皆由人
千般恩怨 一果一報
前因咪羨人

千般恩怨 一果一報
前因咪羨人

\item 鱷魚淚

成功的歡呼到處為他追隨
人生的好境到處與他暢聚
誰知巔峰的他 誰知頂尖的他
悠悠流下兩行淚

是否他心中永遠抹不去恐懼
是否他心中永遠太多掛慮
是否他心中所思 迷惘得不知所止
茫然流下滿眼淚

*是否萬千感慨 從此迷失意義
    茫茫然滿眼眼淚 是喜抑悲
    什麼意義 什麼意義

#寧願終此生你我找得安然
    唔想一生走到尾最終有淚
    成功的幾多得失 難補他一世抑鬱
    和茫然那滿眼淚

\item 強人
是與非 如何分對錯 恨與哀 誰人解因果
敵友之間紛爭 難為彼我 恨愛可有界線 同是分不清楚 
看世間 成敗轉眼就過 弱者 強人都犧牲多 
莫記此中得失 不記恨愛相纏 只記共你當年 曾經相識過


\item 倚天屠龍記
情義繞心中有幾多重 仇恨又卻是誰所種
情仇兩不分 愛中偏有恨 恩怨同重

忘情棄愛世上有真英雄 常人只許讓愛恨纏心中
難忘你恩深 我偏偏有恨 相思難共

情如天 萬里廣闊 仇如海 百般洶湧
要共對亦難 分也不可 愛恨填胸

迷迷惘惘苦痛有幾千重 愁愁怨怨待那日才可終
屠龍刀 倚天劍斬不斷 心中迷夢
屠龍刀 倚天劍斬不斷 心中迷夢

情如天 萬里廣闊 仇如海 百般洶湧
要共對亦難 分也不可 愛恨填胸

迷迷惘惘苦痛有幾千重 愁愁怨怨待那日才可終
屠龍刀 倚天劍斬不斷 心中迷夢
屠龍刀 倚天劍斬不斷 心中迷夢

\item 誓要入刀山
誓要去 入刀山 浩氣壯 過千關
豪情無限 男兒傲氣 地獄也獨來獨往返
存心一闖虎豹穴今朝去幾時還
奈何難盡歡千日醉 此刻相對恨晚

願與你 盡一杯 聚與散 記心間
毋忘情義 長存浩氣 日後再相知未晚
存心一闖虎豹穴今朝去幾時還
奈何難盡歡千日醉 此刻相對恨晚

願與你 盡一杯 聚與散 記心間
毋忘情義 長存浩氣 日後再相知未晚
豪情無限 男兒傲氣 日後再相知未晚

\item 情未了
無限愛盡成輕煙 一朝消散似場夢
情未了緣份已盡 再不見你影蹤

緣盡了就如輕煙 幾多恩愛化為夢
留下我獨來追憶 淚向心中湧

匆匆 太匆匆 再回頭一切成夢
枉我痴 枉我愛 萬縷相思有何用

難望你共諧今生 千番追悔也無用
唯望你我在他生 共對再會重逢

\item 奮鬥
無論歷盡幾次浪
無論受盡多少風霜
無論再要奮鬥幾次
才共你到得彼岸

為你實現萬千美夢
令你樣樣事心願能償
寧願奮鬥到百千次
創出幸福快樂鄉

同我兩手相牽
發千分熱千分光
燃亮着我的愛
為你照前方

同你披荊斬棘
為你衝破前途路障
獻出千般愛心與痴情
一切都奉上


\item 天虹
萬紫千紅 人世萬千癡癡情濃
耀眼金黃 塵俗時刻苦苦追蹤
那青與藍 原來是塵世種種抑鬱
大千色相 幻化彩虹 浮現那天空中
見到天虹 雲際現出光彩重重
令我忽然 明白人生的匆匆
瞬息光華 凝滿萬般千般艷麗
大千色相 艷似天虹
還是會消失影蹤 
輕風吹過 就見天虹 
影蹤被風送

大千色相 艷似天虹
還是會消失影蹤
輕風吹過 就見天虹
影蹤被風送

\item 月光像姐兒
獨:月光高在天空上哎
合:月光像姐兒一般漂亮哎
獨:心裡便嘅光輝又怕哥未知以望見
啊嘥咗光芒呀哎地喲
合:月亮光光在天空上 照見地上小姑娘
獨:未見倒哥哥面 哥哥呀幾時到心上
合:啦
獨:月光有情先至亮哎
合:未似姐情光千丈哎
獨:心裡便嘅相思就怕哥你未知人未見
啊天天盼望呀哎地喲
合:月亮光光天空上 照見姐姐老爹娘
獨:原來哥哥在討親呀擇日免姐盼望
合:啦 月光高在天空上哎
獨:心裡便嘅光輝又怕哥你未知難望見
啊嘥咗光芒呀哎地喲 月亮光光天空上
照見地上小姑娘
獨:未見倒哥哥面 哥哥呀幾時到心上
合:月亮光光天空上 照見姐姐老爹娘
獨:原來哥哥在 討親呀擇日日免姐盼望
合:啦

\item 抉擇
\item 獅子山下
\item 楚留香
\item 女人三十
\item 風雲
\item 上海灘
\item 親情
\item 追憶當日愛
\item 發現灣
\item 仁者無敵
\item 輪流傳
\item 萬般情
\item 勇往直前
\item 女中豪傑
\item 用愛將心偷
\item 情枷愛鎖
\item 龍仇鳳血
\item 衝擊
\item 情債
\item 上海灘龍虎鬥
\item 難忍別離淚
\item 做人愛自由
\item 飲勝
\item 愛在心內暖
\item 四季情
\item 烽火飛花
\item 長為曙光盼
\item 留下你的夢
\item 凱旋歌
\item 未了情
\item 寰宇心聲
\item 楊門女將
\item 遊山樂
\item 千王群英會
\item 想到我就做
\item 快樂自由歌
\item 富貴榮華
\item 命運
\item 倆忘煙水裡(國、粵)
\item 付上千萬倍
\item 萬水千山縱橫
\item 情愛幾多哀
\item 星宿老仙
\item 當我人去後
\item 忘盡心中情
\item 星夜星塵
\item 今晚夜
\item 少女心
\item 人生曲
\item 勇敢的中國人
\item 伴你過一世
\item 心債
\item 胸懷大志
\item 愛定你一個
\item 為你做夢
\item 巾幗英雄
\item 碧海青天
\item 織個幽夢
\item 一生有意義
\item 桃花開
\item 肯去承擔愛
\item 世間始終你好
\item 千愁記舊情
\item 交出我的心
\item 不信愛有罪
\item 冤鬼再見
\item 同途萬里人
\item 夢裡幾番哀
\item 不見我淚流
\item 始終會行運
\item 我與你、他與我
\item 雪山飛狐
\item 心語
\item 大步上青雲
\item 柔情似水
\item 萬里心連心
\item 八陣圖
\item 香港香港
\item 留下我美夢
\item 風裡不回顧
\item 劍伴誰在
\item 深山唱我歌
\item 心隨流水遠
\item 大地情
\item 昂首向天唱
\item 胡漢夢
\item 人類的錯
\item 少年心
\item 傻夢
\item 未了緣
\item 情在昨日
\item 揮出一片心
\item 珍惜這一刻
\item 浪滔滔
\item 交上知心友
\item 還有明天
\item 你我的奧運聖火
\item 江山如此多嬌
\item 做人做到底
\item 投入生命
\item 為自己當好漢
\item 我不應該愛上的人
\item 活著日日看桃花(暫名)
\item 願世間有青天
\item 絕世絕招
\item 人生一切一身擔
\item 笑看風雲
\item 浪蕩江山
\item 梅花烙
\item 梅花三弄之紅塵故事
\item 心箭
\item 江山如此多FUN
\item 做對繽紛戀人
\item 無悔愛你一生
\item 爸爸媽媽陪我放暑假
\item 一生拖你手
\item 留香曲
\item 男兒著眼天地間
\item 真情細說
\item 對一對
\item 十三密殺令
\item 精彩故事
\item 問江山
\item 為了每天三餐
\item 難得有你
\item 圓月下你來依我
\item 無盡是大地
\item 頭頂一片天
\item 清風不染
\item 笑喇笑喇笑一笑
\item 愛你不是渾閒事
\item 一個人
\item 大時代過客
\item 中國人無謂再鬥串
\item 男兒再不負深情
\item 萬里白雲萬里路
\item 霓裳羽衣
\item 清平調
\item 長恨歌
\item 天主始終係愛
\item 做個正覺自在人
\item 萬里天風伴我飛
\item 今天情
\item 美麗緣份
\item 真愛是苦味
\end{itemize}
